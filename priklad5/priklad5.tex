\documentclass[12pt]{article}
\usepackage{amsmath}
\begin{document}
\begin{center}
\begin{tabular}{| c | c | c | c |}
\hline
U[V] & C[F] & R[$\Omega$] & $u_{C}(0)[V]$ \\
\hline
20 & 8 & 100 & 5\\
\hline
\end{tabular}\\
\end{center}
Známe U, R, C, $U_{C(0)}$\\
1)Ohmův zákon:\\ I = $\frac{U_{R}}{R}$\\
2)Kirchhochův zákon:\\ U = $U_{R}+U_{C}$\\ $U-U_{C}-U_{R}=0$\\
3)Axiom:\\$u_{C}'=\frac{1}{C}*I$\\ $U_{C}(0)=U_{CP}$\\
1.krok\\
\[
  u_{C}'=\displaystyle\frac{1}{C}*\displaystyle\frac{1}{R}*U_{R}
\]
\[
  U_{R} = U-U_{C}
\]
\[
  u_{C}'=\displaystyle\frac{1}{RC}*(U-u_{C})
\]
Jedná se o diferenciální rovnici 1. řádu\\
\textit{počáteční podmínka:} $u_{C}(0)=U_{CP}$\\
\[
  u_{C}'+\displaystyle\frac{u_{C}}{RC}= \displaystyle\frac{U}{RC}
\]
Charakteristická rovnice:
\[
  \lambda + \displaystyle\frac{1}{RC}=0
\]

\[
  \lambda =-\displaystyle\frac{1}{RC}
\]
Očekávané řešení $u_{C}(t)=K(t)e^{\lambda t} = K(t)e^{-\frac{t}{RC}}$\\

\end{document}
