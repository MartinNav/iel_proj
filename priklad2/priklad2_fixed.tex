\documentclass{article}
\usepackage{amsmath}
\begin{document}
\begin{tabular}{| c | c | c | c | c | c | c |}
  \hline
   U[V] & $R_{1}$ [$\Omega$] & $R_{2}$ [$\Omega$]& $R_{3}$ [$\Omega$]& $R_{4}$ [$\Omega$]& $R_{5}$ [$\Omega$]& $R_{6}$ [$\Omega$]\\
  \hline
  180 & 250 & 315 & 615 & 180 & 460 & 120\\
  \hline
\end{tabular}\\
\\
Vyřešíme za využití Theveninovy věty.\\
\[
  R_{23} = R_{2} + R_{3} = 315 + 615 = 930 \Omega
\]
\[
  R_{45} = R_4 + R_5 = 180 + 460 = 640 \Omega
\]
\[
  R_{EKV} = R_1 + \displaystyle\frac{R_{23}R_{45}}{R_{23}+R_{45}}
  = 250 + \displaystyle\frac{930 \cdot 640}{930 + 640}
  = 629.1082802547771 \Omega
\]
\[
  I = \displaystyle\frac{U}{R_{REKV}}
  = \displaystyle\frac{180}{629.1082802547771}
  = 0.28611926698390194A
\]
\[
  U_{R45} = U - (I R_1)
  = 180 - (0.28611926698390194\cdot 250)
  = 108.47018325402452 V
\]
\[
  I_{R45} = \displaystyle\frac{U_{R45}}{R_{45}}
  = \displaystyle\frac{108.47018325402452}{640}
  = 0.1694846613344133 A
\]
\[
  U_i = I_{R45} R_5
  = 0.1694846613344133 \cdot 460
  = 77.96294421383013 V
\]
\[
  R_i = \displaystyle\frac{(\displaystyle\frac{R_{23}R_{1}}{R_{23}R_1}+R_4)R_5}{
    \displaystyle\frac{R_1 R_{23}}{R_1 + R_{23}}+R_{45}
  }
  = \displaystyle\frac{(\displaystyle\frac{930\cdot 250}{930+250}+180)\cdot 460}{
    \displaystyle\frac{250 \cdot 930}{250 + 930}+640
  }
  = 207.20259188012557 \Omega
\]
\end{document}
