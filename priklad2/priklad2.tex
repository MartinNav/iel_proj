\documentclass{article}
\usepackage{amsmath}
\begin{document}
\begin{tabular}{| c | c | c | c | c | c | c |}
  \hline
   U[V] & $R_{1}$ [$\Omega$] & $R_{2}$ [$\Omega$]& $R_{3}$ [$\Omega$]& $R_{4}$ [$\Omega$]& $R_{5}$ [$\Omega$]& $R_{6}$ [$\Omega$]\\
  \hline
  180 & 250 & 315 & 615 & 180 & 460 & 120\\
  \hline
\end{tabular}\\
\\
\textbf{Krok 1} - [TODO: write description]\\
\[
  R_{23} = R_{2} + R_{3} = 315 + 615 = 930 \Omega
\]
\[
  R_{123} = \displaystyle\frac{R_{1} \times R_{23}}{R_{1} + R_{23}}=\displaystyle\frac{180 \times 930}{180 + 930} = 150,81081081081 \Omega 
\]
\[
  R_{1234} = R_{123} + R_{4} = 150,81081081081 + 180 = 330,81081081081 \Omega
\]
\[
  R_{i} = \displaystyle\frac{R_{5} \times R_{1234}}{R_{5} + R_{1234}} = \displaystyle\frac{460 \times 330,81081081081}{460 + 330,81081081081} = 192.4265208475732 \Omega
\]
$\textbf{Krok 2}$ Vypočítáme $U_{i}$ pomocí $I_{B}$\\
\[
  \begin{pmatrix}
    R_{1}+R_{2}+R_{3} & -R_{2}-R_{3}\\
    -R_{2}-R_{3} & R_{2}+R_{3}+R_{4}+R_{5}+R_{6}\\
  \end{pmatrix}
  \times
  \begin{pmatrix}
    I_{A}\\
    I_{B}\\
  \end{pmatrix}
  =
  \begin{pmatrix}
    U\\
    0\\
  \end{pmatrix}
\]
\end{document}
