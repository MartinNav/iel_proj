\documentclass{article}
\usepackage{amsmath}
\begin{document}
\begin{tabular}{| c | c | c | c | c | c | c | c | c | c |}
  \hline
   $U$[V] & $I_1$ [A] & $I_2$ [A]& $R_{1}$ [$\Omega$]& $R_{2}$ [$\Omega$]& $R_{3}$ [$\Omega$]& $R_{4}$ [$\Omega$]& $R_{5}$ [$\Omega$]\\
  \hline
130 & 0.95 & 0.50 & 47 & 39 & 58 & 28 & 25\\
  \hline
\end{tabular}\\
\[
  G_1 = \displaystyle\frac{1}{R_1}
  = \displaystyle\frac{1}{47}S
\]
\[
  G_2 = \displaystyle\frac{1}{R_2}
  = \displaystyle\frac{1}{39}S
\]
\[
  G_3 = \displaystyle\frac{1}{R_3}
  = \displaystyle\frac{1}{58}S
\]
\[
  G_4 = \displaystyle\frac{1}{R_4}
  = \displaystyle\frac{1}{28}S
\]
\[
  G_5 = \displaystyle\frac{1}{R_5}
  = \displaystyle\frac{1}{25}S
\]
\[
  I_3 = \displaystyle\frac{U}{R_4 + R_5}
  = \displaystyle\frac{130}{28 + 25}
  = \displaystyle\frac{130}{53}A
  = 2.452830188679245A
\]
 Nyní sestavíme rovnici pro jednotlivé uzly
 \[
   A)G_1 U_A + G_2 (U_A - U_B ) = -I_1
 \]

\end{document}
